\subsection{}

Seja $\left(a_k\right)_{k \in \mathbb{N}_0}$ uma progressão geométrica
convergente para 0.

\subsubsection{}

Tendo $\alpha \geq 1$, pretendemos determinar se a série na equação
\ref{eqn:ex2-serie} é convergente.

\begin{equation}
\label{eqn:ex2-serie}
	\sum^{\infty}_{k=0} a_k^{\alpha}
\end{equation}

\paragraph{} Como a progressão geométrica
$\left(a_k\right)_{k \in \mathbb{N}_0}$ converge para 0, podemos concluir que
a razão $r$ da mesma progressão terá de ser $|r| < 1$ ou que $a_0 = 0$, uma
vez que terá a forma $a_k = a_0 \cdot r^k$. Senão fosse dessa forma, a
progressão necessariamente seria divergente.

\paragraph{} Para o caso $a_0 = 0$, seria trivial concluir que a série seria
convergente, e a sua soma seria forçosamente 0.

\paragraph{} Para o caso de $|r| < 1$, comecemos por demonstrar que a
progressão $\left(a_k\right)^{\alpha}_{k \in \mathbb{N}_0}$, convergiria para
0.

\paragraph{} Com $\alpha \geq 1$, a progressão
$\left(a_k\right)^{\alpha}_{k \in \mathbb{N}_0}$, assumiria a forma
$ a_0^{\alpha} \cdot r^{\alpha k}$. Deste modo, porque estamos a assumir que
$|r| < 1$, podemos concluir que a progressão convergiria para 0.

\paragraph{} Uma vez que $r^{\alpha} \leq r \implies |r^\alpha| < 1$, a
série satisfaria as condições do teorema 3, da página 578 do
livro\cite{santos2016calculo}, pelo que concluímos que a série seria
convergente e com um valor determinado.

\subsubsection{}

\paragraph{} Com a progressão $\left(a_k\right)^\alpha_{k \in \mathbb{N}_0}$ a
assumir a forma $a_0^\alpha \cdot r^{\alpha k}$, verificadas as condições do
teorema 3, da página 578 do livro \cite{santos2016calculo}, a soma da série
seria determinada por:

\begin{align*}
	\sum^{\infty}_{k = 0} a_k^\alpha = \frac{a_0^\alpha}{1 - r^\alpha}
\end{align*}



