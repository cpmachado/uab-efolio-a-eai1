\subsection{}

\paragraph{} Seja a sucessão $(a_n)_{n \in \mathbb{N}}$ como definido na
equação \ref{eqn:ex3-suc}, cujos termos são maiores ou iguais a
$-\frac{3}{5}$.

\begin{equation}
\label{eqn:ex3-suc}
	a_n = 
	\begin{cases}
		a_1 = -\frac{2}{5}\\
		a_{n + 1} = \frac{a_n - 3}{6}, \; n \in \mathbb{N}\\
	\end{cases}
\end{equation}

\subsubsection{}

\paragraph{} Comecemos por determinar $a_{n+1} - a_n$.

\begin{align*}
	a_{n+1} - a_n &= \frac{a_n - 3}{6} - a_n\\
	&= \frac{a_n - 6 \cdot a_n - 3}{6}\\
	&= \frac{- 5 \cdot a_n - 3}{6}\\
	&= -\frac{5}{6} \cdot a_n - \frac{1}{2}\\
\end{align*}

\paragraph{} Seja $g: \mathbb{R} \to \mathbb{R}, g \in C^1$, com
$g(a_n) = -\frac{5}{6} \cdot a_n - \frac{1}{2}$, podemos determinar se a
sucessão é crescente, caso $g(a_n) > 0$, para um determinado $a_n$.

\paragraph{} Para tal vamos calcular o zero da função $g(a_n)$:

\begin{align*}
	g(a_n) &= 0\\
	\iff -\frac{5}{6} \cdot a_n - \frac{1}{2} &= 0\\
	\iff -\frac{5}{6} \cdot a_n  &= \frac{1}{2}\\
	\iff -5 \cdot a_n  &= 3\\
	\iff a_n  &= -\frac{3}{5}\\
\end{align*}

\paragraph{} Para $a_n = -\frac{2}{5}$ , temos $g(a_n) = -\frac{13}{30} < 0$,
como $a_n \geq -\frac{3}{5}$, podemos concluir que a sucessão é monótona
decrescente e minorada por $-\frac{3}{5}$.

\newpage

\subsubsection{}

\paragraph{} Dado que a sucessão $a_n$ é monótona e decrescente, e minorada
por $-\frac{3}{5}$, podemos concluir que tanto o ínfimo como o supremo existem
e que são:

\begin{align*}
	\inf{a_n : n \in \mathbb{N}} &= -\frac{3}{5}\\
	\sup{a_n : n \in \mathbb{N}} &= a_0 = -\frac{2}{5}
\end{align*}
