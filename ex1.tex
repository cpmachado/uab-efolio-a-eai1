\subsection{}

\paragraph{} No ramo esquerdo da equação, temos uma série
geométrica, para o qual a razão é $|r| < 1$. Como a razão $r$ verifica a
condição, verificam-se todas as condições para aplicar o teorema 3, da página
578 do livro\cite{santos2016calculo}.
Pelo que podemos calcular a soma da série, com $a_0 = 1$, por

\begin{align*}
	\sum^{\infty}_{k = 0} r^k = \frac{a_0}{1 - r} = \frac{1}{1 - r}.
\end{align*}

\paragraph{Caso base}


Para $m = 0$, verificamos que
\begin{align*}
	\sum^{\infty}_{k=0} r^m = \frac{r^0}{1 - r} = \frac{1}{1 - r}.
\end{align*}

Desta forma, prova-se o caso base.

\paragraph{Hipótese de indução} Para $|r| < 1$, suponhamos que

\begin{equation}
\label{eqn:ex1-hipotese}
	\sum^{\infty}_{k=m} r^k = \frac{r^m}{1 - r}, \; m \in \mathbb{N}_{0}.
\end{equation}

\paragraph{Tese de indução}

\begin{equation}
\label{eqn:ex1-tese}
	\sum^{\infty}_{k=m+1} r^k = \frac{r^{m + 1}}{1 - r}
\end{equation}

\newpage

\paragraph{Passo de indução}

\paragraph{} A partir do ramo esquerdo da tese indução(equação \ref{eqn:ex1-tese}),
vamos passar por uma substituição pela hipótese de indução
(equação \ref{eqn:ex1-hipotese}), assim demonstrando a equivalência por indução.

\begin{align*}
	\sum^{\infty}_{k=m+1} r^k &= \left(\sum^{\infty}_{k=m} r^k\right) - r^m \\
	&= \left(\frac{r^m}{1 - r}\right) - r^m \;
	 \text{(aplica-se a hipótese da equação \ref{eqn:ex1-hipotese})}\\
	&= \frac{r^m}{1 - r} - r^m \frac{1 - r}{1 - r}\\
	&= \frac{r^m - r^m(1 - r)}{1 - r} \\
	&= \frac{r^m - r^m + r^{m+1}}{1 - r} \\
	&= \frac{r^{m+1}}{1 - r}\\
\end{align*}


\paragraph{Conclusão} Pelo método de indução matemática, podemos então
concluir que, para $|r| < 1$, temos

\begin{align*}
	\sum^{\infty}_{k=m} r^k = \frac{r^m}{1 - r}, \; m \in \mathbb{N}_{0}.
\end{align*}
